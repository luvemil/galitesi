%************************************************
\chapter{Teoria}\label{ch:teoria}
%************************************************
\section{Introduzione}

La TDA si prefigge come obiettivo quello di trovare strutture complesse in un insieme di dati. Finora sono stati conseguiti risultati come la determinazione di nuove variabili che influenzano l'attività neurale \cite{Spreemann2015}, la classificazione di traiettorie in robotica \cite{Pokorny2014}, l'identificazione di nuovi tipi di cancro al seno \cite{Lum2013}, e molti altri.

La novità della TDA sta nel provare a catturare la \emph{forma} dei dati e, in questa, cercare proprietà topologiche interessanti che costituiscano un segnale anziché un rumore.

\begin{figure}[h]
  \begin{center}
    \includegraphics[width=.4\linewidth]{gfx/three_clusters_small.pdf}
    \caption{Dati divisi in più cluster}
    \label{fig:clusters}
  \end{center}
\end{figure}

Ad esempio, consideriamo un insieme di dati come in \cref{fig:clusters}. È chiaro a chi osserva che vi sono tre gruppi di punti, tuttavia la formalizzazione matematica di questa osservazione non è immediata.

Il modo usato dalla TDA è l'\emph{omologia persistente}, di cui diamo un'introduzione informale per poi riprenderlo in (NMDC: inserire riferimento al capitolo). Sia
\begin{equation*}
  X=\{x_1,\dots, x_n\}
\end{equation*}
il nostro insieme di dati. Consideriamo per $\varepsilon>0$ l'insieme
\begin{equation*}
  \widehat{X}_\varepsilon=\bigcup_{0\leq i\leq n} B(x_i,\varepsilon)
\end{equation*}
dove $B(x_i,\varepsilon)$ è la palla di centro $x_i$ e raggio $\varepsilon$. Osserviamo che esiste un intervallo di valori $a \leq \varepsilon \leq b$ per cui $\widehat{X}_\varepsilon$ appare come in \cref{fig:clusters_fat}.

\begin{figure}[h]
  \begin{center}
    \includegraphics[width=.4\linewidth]{gfx/three_clusters_fat.pdf}
    \caption{$\widehat{X}_\varepsilon$}
    \label{fig:clusters_fat}
  \end{center}
\end{figure}

Allora possiamo possiamo calcolare il numero di componenti connesse di $\widehat{X}_\varepsilon$, o equivalentemente la dimensione del gruppo di omologia $H_{0}(\widehat{X}_\varepsilon;k)$, dove $k$ è un campo. L'osservazione che $X$ è composto essenzialmente da tre componenti è espressa dal fatto che
\begin{equation*}
  \mathrm{dim}_k(H_{0}(\widehat{X}_\varepsilon;k))=3
\end{equation*}
per un intervallo notevole di valori di $\varepsilon$, e da un certo $\overline{\varepsilon}$ in poi diventa~1.

Ovviamente, non è necessario parlare di dimensione del gruppo di omologia $H_{0}$ per discutere del numero di componenti connesse. Tuttavia, se consideriamo l'insieme di dati $X$ come in \cref{fig:circle}, possiamo chiederci come formalizzare l'intuizione che essi sono disposti in forma circolare.

\begin{figure}[h]
  \begin{center}
    \includegraphics[width=\linewidth]{gfx/statistical_circle.pdf}
    \caption{Campionamento da una corona circolare}
    \label{fig:circle}
  \end{center}
\end{figure}

Ancora una volta possiamo considerare l'insieme $\widehat{X}_\varepsilon$ per diversi valori di $\varepsilon$ come in \cref{fig:circlecomparison} e considerare stavolta il gruppo di omologia $H_{1}(\widehat{X}_\varepsilon;k)$.

\begin{figure}[h]
  \begin{center}
    \includegraphics[width=.7\paperwidth]{gfx/statistical_circle_comparison.pdf}
    \caption{$\widehat{X}_\varepsilon$ al variare di $\varepsilon$}
    \label{fig:circlecomparison}
  \end{center}
\end{figure}

La dimensione di $H_1(\widehat{X}_\varepsilon;k)$ varia fino a stabilizzarsi su 1. Questo ci dice che c'è essenzialmente un buco 1-dimensionale nei dati.

Possiamo anche osservare variazioni di struttura al variare della scala di riferimento. Ad esempio, in \cref{fig:moreclusters} possiamo vedere che i tre cluster sulla sinistra collassano in un unico cluster se condiseriamo $\widehat{X}_\varepsilon$ per $\varepsilon \gtrsim e_2/2$.

\begin{figure}[h]
  \includegraphics[width=.7\paperwidth]{gfx/more_clusters.pdf}
  \caption{Cluster su scale diverse}
  \label{fig:moreclusters}
\end{figure}

Per visualizzare l'andamento di $dim_k(\widehat{X}_\varepsilon)$ al variare di $\varepsilon$ usiamo un tipo di grafico chiamato \emph{persistence barcode}: per ogni \emph{feature} presente nei dati, disegnamo un segmento orizzontale lungo quanto l'intervallo di lunghezze di $\varepsilon$ in cui la feature persiste, come in \cref{fig:moreclusterbarcode}.

\begin{figure}[h]
  \includegraphics[width=.7\paperwidth]{gfx/more_clusters_barcodes.pdf}
  \caption{Un esempio di persistence barcode}
  \label{fig:moreclusterbarcode}
\end{figure}

Il grafico fa interpretato nel seguente modo: all'inizio vi sono 26 punti distinti, al crescere di $\varepsilon$ questi punti vengono uniti ad altri e quindi il numero si riduce, finché per $\varepsilon \gtrsim e_1/2$ restano essenzialmente 4 cluster, da $e_2/2$ ne restano solo due e da $e_3/2$ il gruppo di omologia $H_0(\widehat{X}_\varepsilon;k)$ diventa banale. (NMDC: aggiusta il grafico del barcode)

Possiamo sempre usare questa rappresentazione grazie alla decomposizione dei persistence barcodes garantita dal teorema (NMDC: aggiungere riferimento).

Un altro aspetto che l'omologia persistente cattura è la relazione fra i gruppi di omologia nelle diverse scale, in particolare $H_*(\widehat{X}_\varepsilon;k)$ è funtoriale rispetto all'ordine di $(\R,\leq)$, cioé se $\varepsilon_1 \leq \varepsilon_2$ allora c'è una mappa $H_*(\widehat{X}_{\varepsilon_1};k)\to H_*(\widehat{X}_{\varepsilon_2};k)$ e se $\varepsilon_1\leq\varepsilon_2\leq\varepsilon_3$, allora la mappa associata a $\varepsilon_1\leq\varepsilon_3$ è uguale alla composizione delle due mappe associate a $\varepsilon_1\leq\varepsilon_2$ e $\varepsilon_2\leq\varepsilon_3$.

Questo ci consente di catturare proprietà come quelle che si osservano in \cref{fig:doublecircle}.

\begin{figure}[ht]
  \begin{center}
    \includegraphics{gfx/double_circle_small.pdf}
    \caption{Un doppio anello}
    \label{fig:doublecircle}
  \end{center}
\end{figure}

In \cref{fig:doublecirclecomparison} osserviamo che al variare di $\varepsilon$ la dimensione di $H_1(\widehat{X}_\varepsilon;k)$ resta 1, tuttavia è chiaro che i due anelli sono due proprietà distinte dei dati. Questa distinzione non è racchiusa nel gruppo di omologia $H_1$, mentre la si vede dal fatto che la mappa
\begin{equation*}
H_1(\widehat{X}_{\varepsilon_1};k)\xto{0}H_1(\widehat{X}_{\varepsilon_2};k)
\end{equation*}
associata a $\varepsilon_1\leq\varepsilon_2$ è il morfismo nullo. Questo ci dice che non ci sono relazioni fra i due gruppi di omologia.

\begin{figure}[ht]
  \begin{center}
    \begin{subfigure}[b]{.4\textwidth}
      \includegraphics[width=\textwidth]{gfx/double_circle_medium.pdf}
      \caption{$\widehat{X}_{\varepsilon_1}$}
    \end{subfigure}
    \begin{subfigure}[b]{.4\textwidth}
      \includegraphics[width=\textwidth]{gfx/double_circle_fat.pdf}
      \caption{$\widehat{X}_{\varepsilon_2}$}
    \end{subfigure}
    \caption{Variazione delle proprietà di $\widehat{X}_\varepsilon$}  \label{fig:doublecirclecomparison}
  \end{center}
\end{figure}

(NMDC:dire qualcosa sull'algoritmo Mapper?)

Nel resto del capitolo ci occuperemo di costruire l'omologia persistente, con attenzione all'aspetto computazionale.

\clearpage

\section{Omologia simpliciale}

\begin{sloppypar}
  Fissato un campo $k$, ad ogni spazio topologico possiamo associare una successione di $k$-spazi vettoriali $H_i(X;k)$ detti \emph{gruppi di omologia}. Per la precisione si tratta di una successione di funtori ${H_*(-;k):\Top \to \Vectk}$. Nel resto della sezione ci occuperemo di definire in modo operativo questi gruppi.
\end{sloppypar}

Esistono diversi modi di calcolare i gruppi di omologia. Il più generale e potente è l'omologia singolare, tuttavia questa risulta scomoda da usare in pratica perché richiede di lavorare con quozienti di spazi vettoriali di dimensione più che numerabile. Per aggirare il problema definiremo soltanto l'omologia simpliciale.

In virtù degli assiomi di Eilenberg-Steenrod \cite{Eilenberg1945, Eilenberg} le due definizioni sono equivalenti (almeno per gli spazi che andremo a considerare). Per una trattazione più dettagliata si vedano \cite{Hatcher2015} o \cite{Rotman1988}.

Si procederà associando all'insieme di dati $X$ uno spazio topologico (detto complesso simpliciale) per ogni parametro $\varepsilon \in \R^+$ e si definirà l'omologia solo per questi particolari spazi topologici.

Gli spazi $\widehat{X}_\varepsilon$ usati nell'introduzione, sebbene comodi per introdurre un'idea intuitiva di persistenza, non sono l'ambiente naturale in cui lavorare, quindi non verranno più usati nella trattazione formale. Si osservi, però, che l'omologia di $\widehat{X}_\varepsilon$ è equivalente all'omologia del complesso simpliciale di \v{C}ech di parametro $\varepsilon$ associato a $X$. Per questioni di comodità, tuttavia, noi lavoreremo principalmente con il complesso di Vietoris-Rips.

\subsection{Complessi simpliciali}

\begin{sloppypar}
  I complessi simpliciali sono particolari spazi topologici che hanno una descrizione combinatorica. Dato un insieme di punti ${S=\{s_0,\dots,s_n\}}$ in $\R^k$ diremo che sono in posizione generale se non sono contenuti in nessun sottospazio affine di dimensione minore di $n$. Se $S$ è in posizione generale, il suo inviluppo convesso $\sigma(S)$ è detto \emph{($n$-)simplesso generato da $S$}. I punti $s_i$ di $S$ si chiamano \emph{vertici} di $\sigma(S)$. Se $\emptyset\neq T$ è un sottinsieme di $S$, $\sigma(T)$ è detto \emph{faccia} di $\sigma(S)$.
\end{sloppypar}

Con questi ingredienti possiamo dare la seguente
\begin{defn}
Un \emph{complesso simpliciale} (finito) $K$ è una famiglia finita di simplessi in uno spazio euclideo tali che:
\begin{enumerate}
  \item Se $\sigma\in K$ e $\tau$ è una faccia di $\sigma$, allora $\tau \in K$.
  \item Se $\sigma,\tau\in K$, allora il simplesso $\sigma\cap\tau$ è una faccia sia di $\sigma$ sia di $\tau$.
\end{enumerate}
\end{defn}

\`E chiaro che un complesso simpliciale è determinato essenzialmente da proprietà combinatoriche dell'insieme dei suoi vertici, che motiva la seguente costruzione astratta.

\begin{defn}
  Un \emph{complesso simpliciale astratto} $X$ è il dato della coppia $(V(X), \Sigma(X))$, dove $V(X)$ è un insieme finito, i cui elementi sono i \emph{vertici} di $X$, e $\Sigma(X)$ è una famiglia di sottinsiemi non vuoti di $V(X)$, i cui elementi sono detti \emph{simplessi} di $X$, tale che:
  \begin{enumerate}
    \item se $v\in V(X)$ allora $\{v\}\in\Sigma(X)$, e
    \item se $\sigma \in \Sigma(X)$ e $\emptyset\neq\tau\subseteq\sigma$, allora $\tau\in\Sigma(X)$.
  \end{enumerate}

  Un simplesso $\sigma\in\Sigma(X)$ è detto $q$-simplesso se $|\#\sigma| = q+1$. Indichiamo con $X_q$ l'insieme dei $q$-simplessi di $X$.
\end{defn}

\begin{rmk}
  Ogni complesso simpliciale $K$ determina un complesso simpliciale astratto $\widehat{K}$ tale che $V(\widehat{K})$ è l'insieme dei vertici dei simplessi di $K$ e un sottinsieme di $V(\widehat{K})$ è in $\Sigma(\widehat{K})$ se e solo se è l'insieme dei vertici di un simplesso di $K$.
\end{rmk}

Possiamo anche definire i morfismi $f:X\to Y$ fra complessi simpliciali astratti come le mappe $f_V:V(X)\to V(Y)$ fra i sottostanti insiemi di vertici e tali che $f_V(\sigma)\in \Sigma(Y)$ per ogni $\sigma \in \Sigma(X)$.

(NMDC: inserire qualche disegno, eventualmente un riferimento a qualche testo sugli oggetti simpliciali)

Ad ogni complesso simpliciale astratto $X$ si può associare un complesso simpliciale $|X|$ detto la \emph{realizzazione geometrica} di $X$ (NMDC: aggiungere un riferimento) e tale che i morfismi $f:X\to Y$ siano mandati in mappe continue $|f|:|X|\to |Y|$ in maniera funtoriale, cioé $|g\circ f|=|g|\circ |f|$. Inoltre, ogni complesso simpliciale $K$ è omeomorfo alla realizzazione geometrica del complesso simpliciale astratto $\widehat{K}$ ad esso associato.

\subsection{Omologia simpliciale}

Ora definiremo i gruppi di omologia associati a un complesso simpliciale astratto.

\begin{defn}
  Sia $k$ un gruppo o un campo o un anello e
  sia $X=(V(X),\Sigma(X))$ un complesso simpliciale astratto, insieme con un ordine totale sull'insieme $V(X)$. Il gruppo dei \emph{$q$-cicli} di $X$ su $k$ è il $k$-modulo $C_q(X,k)$ generato dagli elementi
  \begin{itemize}
    \item $[v_0,\dots, v_q]$ con $v_0,\dots, v_q\in V(X)$ e tali che $\{v_0,\dots,v_q\}\in\Sigma(X)$
  \end{itemize}
  modulo le seguenti relazioni:
  \begin{enumerate}
    \item $[v_0,\dots,v_q]=0$ se $v_i = v_j$ per qualche $i\neq j$,
    \item $[v_{\sigma(0)},\dots,v_{\sigma(q)}]=sign(\sigma)[v_0,\dots,v_q]$ per ogni permutazione $\sigma$ di $\{0,\dots,q\}$.
  \end{enumerate}

  Scriveremo $C_*(X,k)$ per indicare tutti i gruppi dei cicli in tutti i gradi, eventualmente sottintendendo il campo $k$.
\end{defn}

\begin{rmk}
  Nella definizione precedente si è tenuta traccia dell'ordine dei vertici dei complessi simpliciali perché esso è importante ai fini dell'omologia, dunque d'ora in poi consideriamo sempre i complessi simpliciali ordinati.
\end{rmk}

\begin{lemma}\label{lemma:sollcicli}
  Dati due complessi simpliciali astratti $X$ e $Y$ e un morfismo $f$ fra essi, allora $f$ induce un omomorfismo $f_q:C_q(X)\to C_q(Y)$ per ogni $q\in\N$ definito da:
  \begin{align*}
    f_q:C_q(X)\longrightarrow & C_q(Y)\\
    [v_0,\dots,v_q] \mapsto & [f(v_0),\dots,f(v_q)].
  \end{align*}

  Scriveremo $f_*:C_*(X)\to C_*(Y)$ per indicare tutti questi morfismi.
\end{lemma}

\begin{rmk}
  Il motivo per cui il gruppo dei $q$-cicli è stato introdotto mediante generatori e relazioni è che in questo modo è più facile sollevare mappe di complessi simpliciali astratti a omomorfismi fra i gruppi dei cicli. Se avessimo semplicemente definito il gruppo dei $q$-cicli come il $k$-modulo generato da $X_q$ sarebbe stato necessario modificare la definizione di $f_*$ in modo che i cicli mappassero correttamente, perdendo notevolmente in eleganza.
\end{rmk}

\begin{defn}
  Sia $X$ un complesso simpliciale astratto, allora esistono morfismi $\partial^X_q:C_q(X) \to C_{q-1}(X)$ per ogni $q\in\N_0$ definiti come:
  \begin{equation*}
    \partial_q([v_0,\dots,v_q])=\sum_{i=0}^q (-1)^i[v_0,\dots,\widehat{v}_i,\dots,v_q]
  \end{equation*}
  dove la notazione $\widehat{v}_i$ significa che l'$i$-mo elemento non è presente. Scriveremo $\partial$ invece che $\partial^X$ quando non è necessario specificare il complesso simpliciale sottostante.
\end{defn}

\begin{lemma}
  Nelle notazioni precedenti, $\partial_{q-1}\circ \partial_q=0$.
\end{lemma}

\begin{rmk}
  Il lemma precedente ci dice che per ogni complesso simpliciale astratto $X$ i suoi cicli $C_*(X)$ formano un complesso di $k$-moduli. Allora data una mappa di complessi simpliciali astratti $f:X\to Y$, il suo sollevamento $f_*:C_*(X)\to C_*(Y)$ è un morfismo di complessi di $k$-moduli, cioé $f_{q-1}\circ \partial^X_q=\partial^Y_q\circ f_q$. Quest'ultima proprietà può essere espressa dicendo che il seguente diagramma è commutativo (cioé che qualsiasi percorso si segua componendo i morfismi non cambia il risultato della composizione).
  \begin{equation*}
    \begin{tikzcd}
      C_q(X)\arrow[r,"\partial^X_q"]\arrow[d,"f_q"]
        &C_{q-1}(X)\arrow[d,"f_{q-1}"]\\
      C_q(Y)\arrow[r,"\partial^Y_q"]&C_{q-1}(Y)
    \end{tikzcd}
  \end{equation*}

  Possiamo visualizzare $C_*(X)$ come la seguente successione:
  \begin{equation*}
    \begin{tikzcd}
      \dots \arrow{r}&
        C_2\arrow{r}{\partial_2}&
          C_1\arrow{r}{\partial_1}&
            C_0
    \end{tikzcd}
  \end{equation*}
  a cui per completezza aggiungeremo il morfismo $\partial_0:C_0\to 0$.
\end{rmk}

\begin{defn}
  Dato un complesso impliciale astratto $X$, per ogni $q\in\N_0$ vale $Im\,\partial_q \subseteq Ker\,\partial_{q-1}$. Definiamo il $q$-mo gruppo di omologia di $X$ come il quoziente
  \begin{equation*}
    H_q(X) = \frac{Ker\,\partial_q}{Im\,\partial_{q+1}}.
  \end{equation*}
\end{defn}

Il seguente esempio serve a mostrare intuitivamente come l'omologia è collegata alle proprietà geometriche di un complesso simpliciale. Consideriamo il complesso simpliciale astratto $X=(V(X),\Sigma(X))$ con:
\begin{align*}
  V(X) &= ( A,B,C,D,E )\\
  \Sigma(X) &= \{ \{A\}, \{B\}, \{C\}, \{D\}, \{E\}, \{A,B\}, \{C,D\}, \{D,E\}, \{C,E\} \}
\end{align*}
rappresentato in figura \cref{fig:simplicialcomplex}. La scrittura $(A,\dots,E)$ indica che l'ordine è $A<\dots<E$.

\begin{figure}[h]
  \includegraphics[width=.7\linewidth]{gfx/example_homology.pdf}
  \caption{Un complesso simpliciale}
  \label{fig:simplicialcomplex}
\end{figure}

Allora $C_0$ è lo spazio vettoriale generato da $[A], [B], [C], [D], [E]$ e $C_1$ da $[A,B], [C,D], [D,E], [C,E]$.
Applicando il differenziale $\partial_1$ ai generatori di $C_1$ otteniamo:
\begin{align*}
  \partial_1([A,B]) &= [B] - [A]\\
  \partial_1([C,D]) &= [D] - [C]\\
  \partial_1([D,E]) &= [E] - [D]\\
  \partial_1([C,E]) &= [E] - [C]
\end{align*}

Per semplicità, si può anche scrivere $\partial_1$ in forma matriciale come
\begin{equation*}
  \bordermatrix{
     & [A,B] & [C,D] & [D,E] & [C,E]\cr
    [A] & -1 & 0 & 0 & 0 \cr
    [B] & 1 & 0 & 0 & 0 \cr
    [C] & 0 & -1 & 0 & -1 \cr
    [D] & 0 & 1 & -1 & 0\cr
    [E] & 0 & 0 & 1 & 1\cr
  },
\end{equation*}
da cui risulta evidente che $Im\,\partial_1$ ha dimensione 3, quindi $H_0(X) = Ker\,\partial_0 / Im\,\partial_1$ è un $k$-spazio di dimensione $2$. Si osservi che $X$ ha esattamente 2 componenti connesse.

Analogamente $Ker\,\partial_1$ ha dimensione 1 e $C_2$ è lo spazio banale, quindi $H_1(X)$ ha dimensione 1, che rappresenta il fatto che c'è un loop nella spezzata $[C,D] + [D,E] + [E,C]$ (infatti $Ker\,\partial_1$ è generato proprio da questo vettore).

\begin{figure}[h]
  \includegraphics[width=.7\linewidth]{gfx/example_homology_loop.pdf}
  \caption{Il complesso $X$ con l'aggiunta di un $2$-simplesso}
  \label{fig:simplicialcomplexloop}
\end{figure}

Se aggiungiamo a $X$ il simplesso $\{C,D,E\}$ come in \cref{fig:simplicialcomplexloop} abbiamo che $C_2$ è generato da $[C,D,E]$ e $\partial_2([C,D,E]) = [D,E] - [C,E] + [C,D] = [C,D] + [D,E] + [E,C]$. Quindi chiudendo il loop con un simplesso abbiamo che
$Im\,\partial_2 = Ker\,\partial_1$ e $H_1(X)$ è lo spazio banale.

\begin{rmk}
  L'assegnazione di $H_*(X)$ a un complesso simpliciale astratto $X$ è in realtà un funtore, cioé per ogni morfismo di complessi simpliciali astratti $f:X\to Y$ esiste un omomorfismo di $k$-spazi vettoriali $f_*:H_*(X)\to H_*(Y)$ (e in realtà $f_*$ è il morfismo indotto da $f_*:C_*(X)\to C_*(Y)$) e $(g\circ f)_* = g_* \circ f_*$.
\end{rmk}

\section{Omologia Persistente}

Nella sezione precedente è stata definita l'omologia simpliciale e si è visto con alcuni esempi come questa sia collegata a proprietà topologiche del complesso simpliciale in analisi. Quello che resta per poter usare l'omologia per studiare una nube di punti è un modo di costruire un simplesso simpliciale a partire dai dati, il che essenzialmente si riduce a studiare i dati ad una certa scala di grandezza. Tuttavia, non è necessariamente ovvio come fare e, come si è visto nell'introduzione, talvolta non esiste una scelta di scala che consenta di catturare tutte le proprietà (omologiche) dei nostri dati.

Per questo motivo si introduce il concetto di persistenza, che è un modo di controllare diverse scale contemporaneamente e le relazioni tra queste.

\begin{defn}
  Un \emph{insieme persistente} è una famiglia di insiemi $\{X_r\}_{\R}$ indicizzata da $\R$ e tale che per ogni coppia di numeri reali $r<s$ esiste una funzione $f_{s,r}:X_r\to X_s$ e queste funzioni sono tali che se $r< s< t$ allora $f_{t,r} = f_{t,s}\circ f_{s,r}$. Più in generale un oggetto persistente in una categoria $\mathcal{C}$ è un funtore $(R,\leq)\to \mathcal{C}$, quindi si può parlare di \emph{spazi vettoriali persistenti},
  \emph{spazi topologici persistenti}, \emph{complessi simpliciali persistenti}, ecc.
\end{defn}

A questo punto è possibile associare un complesso simpliciale astratto ai nostri dati nel seguente modo.

\begin{defn}
  Sia $X$ uno spazio metrico finito. Fissato un numero reale positivo $r$ il \emph{complesso di Vietoris-Rips} di $X$ è
  un complesso simpliciale astratto $VR(X,r)$ così definito:
  \begin{itemize}
    \item l'insieme dei vertici di $VR(X,r)$ è $X$,
    \item la collezione $\{x_0,\dots,x_n\}\subseteq X$ è un simplesso di $VR(X,r)$ se e solo se
    \begin{equation*}
      d(x_i,x_j) \leq r \mathrm{\text{ per tutti gli }}i,j\in\{0,\dots,n\}.
    \end{equation*}
  \end{itemize}
\end{defn}

\begin{rmk}
  Se $r<s$ esiste un'ovvia iniezione $VR(X,r)\to VR(X,s)$, siccome gli insiemi dei vertici coincidono e ogni simplesso di $VR(X,r)$ è anche un simplesso di $VR(X,s)$, quindi la famiglia $\{VR(X,r)\}_{r\in\R}$ forma un complesso simpliciale persistente.
\end{rmk}

\begin{figure}[h]
  \includegraphics[width=.7\linewidth]{gfx/exhagon.pdf}
  \caption{Esempio di complesso di Vietoris-Rips}
  \label{fig:exhagon}
\end{figure}

Ad esempio si consideri un esagono regolare come in \cref{fig:exhagon}. Se $R$ è la misura del lato dell'esagono, le tre figure rappresentano $VR(X,r)$ rispettivamente con $0\leq r < R$, $R\leq r < \sqrt{3}R$, e $r \geq \sqrt{3}R$. I complessi di Vietoris-Rips sono elementari nelle prime due figure. Si osservi, invece, che nella terza vi sono 8 $2$-simplessi, sebbene intuitivamente sia sufficiente usarne 4 per vedere che lo spazio diventa banale a quella scala.

\begin{sloppypar}
  A questo punto applicando $H_*(-,k)$ a $\{VR(X,r)\}_{\R}$ otteniamo una famiglia di spazi vettoriali $\{H_*(VR(X,r),k)\}$ che insieme alle mappe ${H_*(f_{s,r},k):H_*(VR(X,r))\to H_*(VR(X,s))}$ formano uno spazio vettoriale persistente. Osserviamo inoltre che essendo partiti da uno spazio metrico finito $X$ ci saranno solo un numero finito di complessi di Vietoris-Rips al variare di $r$.
\end{sloppypar}

Gli spazi $\{H_*(VR(X,r),k)\}_{r\in\R}$ racchiudono l'informazione topologica dello spazio metrico $X$, tuttavia questa definizione astratta potrebbero sembrare di difficile utilizzo: assumendo di poter calcolare tutti gli $H_q(VR(X,r),k)$ per tutti gli $r$ e per piccoli valori di $q$, la domanda successiva diventa cosa si può dire delle interazioni fra i vari complessi su scale diverse?

\begin{sloppypar}
  In linea di principio dobbiamo considerare per ogni ${f_{s,r}:VR(X,r)\to VR(X,s)}$ non banale l'associato morfismo di spazi vettoriali ${H_q(f_{s,r}):H_q(VR(X,r))\to H_q(VR(X,s))}$ e tenere traccia delle dimensioni degli spazi $H_q(VR(X,r))$, $H_q(VR(X,s))$, e $Im\,H_q(f_{s,r})$.
\end{sloppypar}

Al fine di poter meglio comprendere come sono fatti i gruppi di omologia persistenti, studiamo meglio gli spazi vettoriali persistenti. D'ora in poi per semplicità diremo che $V$ è uno spazio vettoriale persistente invece che indicare esplicitamente tutta la famiglia $\{V_r\}$. Qualora fosse necessario esplicitare i morfismi $f_{s,r}:V_r\to V_s$ li indicheremo con la notazione $(V,f)$.

\begin{defn}
  Sia $I\subseteq \R$. Diremo che $I$ è un intervallo di $\R$ se dati comunque $r,s\in I$ tali che $r<s$ e $t$ tale che $r<t<s$, allora anche $t\in I$. Definiamo \emph{moduli intervallo} gli spazi vettoriali persistenti $k_I$, dove $I$ è un intervallo di $\R$, tali che $k_I(r) = k$ se $r\in I$ e $0$ altrimenti.
\end{defn}

\begin{defn}
  Si definisce \emph{somma diretta} di due spazi vettoriali persistenti $(V,f^V)$ e $(W,f^W)$ lo spazio vettoriale persistente $U$ definito come $U_r = V_r\oplus W_r$ per ciascun $r$ e $f^U_{s,r} = f^V_{s,r}\oplus f^W_{s,r}$, e lo si indica come $(V\oplus W,f^V\oplus f^W)$.
\end{defn}

\begin{figure}[ht]
  \begin{center}
    \begin{subfigure}[b]{.4\textwidth}
      \includegraphics[width=\textwidth]{gfx/barcodes_single.pdf}
      \caption{$k_I$}\label{fig:barcodes:single}
    \end{subfigure}
    \begin{subfigure}[b]{.4\textwidth}
      \includegraphics[width=\textwidth]{gfx/barcodes_multiple.pdf}
      \caption{$\bigoplus_{I\in\mathcal{D}} k_I$}\label{fig:barcodes:multiple}
    \end{subfigure}
    \caption{Esempi di codici a barre}  \label{fig:barcodes}
  \end{center}
\end{figure}


Poiché i moduli intervallo sono associati a intervalli di $\R$ li si può rappresentare come barre lunghe quanto l'intervallo ad essi associati (\cref{fig:barcodes:single}), e similmente una somma diretta finita di moduli intervallo
$k_{I_1}\oplus k_{I_2}\oplus\cdots \oplus k_{I_n}$ può essere rappresentata con quelli che sono detti codici a barre (\cref{fig:barcodes:multiple}). Il seguente teorema ci dice che in realtà tutti gli spazi vettoriali persistenti si posso rappresentare in questo modo.

\begin{thm}[Decomposizione degli spazi vettoriali persistenti \cite{Crawley-Boevey2012}]
  Sia $(V,f)$ uno spazio vettoriale persistente tale che $V_r$ ha dimensione finita per ogni $r\in\R$, allora esiste un
  multi-insieme $\mathcal{D}$ di intervalli, cioé una famiglia di intervalli con ripetizioni, tale che
  \begin{equation*}
    V\cong \bigoplus_{I\in\mathcal{D}} k_I.
  \end{equation*}
\end{thm}
