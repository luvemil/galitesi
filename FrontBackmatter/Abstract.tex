%*******************************************************
% Abstract
%*******************************************************
%\renewcommand{\abstractname}{Abstract}
\pdfbookmark[1]{Sommario}{Sommario}
\begingroup
\let\clearpage\relax
\let\cleardoublepage\relax
\let\cleardoublepage\relax

\chapter*{Sommario}

Lo studio dei Big Data è più che mai attuale vista la grande quantità di dati che viene prodotta dai più disparati settori delle scienze applicate e dell'industria, e vi è sempre più necessità di tecniche di analisi di tipo qualitativo, oltre che quantitativo.

Da un lato vi è stato lo sviluppo di strumenti di analisi estremamente generali, come quelle tecniche che insieme vengono chiamate Machine Learning, per i quali esistono pacchetti software estremamente moderni che possono essere usati in ambiente production, oltre che per la ricerca. Dall'altro, l'efficienza dei moderni processori ha consentito l'applicazione di queste tecniche ad ampio spettro su enormi quantità di dati non strutturati.

Un nuovo tipo di problema, tuttavia, è emerso: ora che abbiamo a disposizione tutti questi dati non strutturati, che spesso sono collezioni di punti in spazi vettoriali di dimensione elevata, vorremmo capirne la struttura globale.

Le tecniche attualmente utilizzate tendono a non essere sufficientemente sensibili alla struttura dei dati o a non individuare con accuratezza le loro proprietà geometriche. Per risolvere questo problema sono state sviluppate una serie di tecniche, che insieme vengono chiamate Topological Data Analysis (TDA), alla cui base c'è l'omologia persistente, che noi ci occuperemo di presentare.

L'omologia persistente consente di studiare gli invarianti topologici dei dati \emph{su tutte le scale} allo stesso tempo, e può essere usata da sola per scoprire strutture all'interno dei dati, ad esempio componenti connesse o presenza di buchi ($n$-dimensionali), o in congiunzione con tecniche di Machine Learning come le (\emph{Support Vector Machines}), mediante la definizione di opportuni kernel nello spazio dei diagrammi persistenti.

\vfill

\endgroup

\vfill
