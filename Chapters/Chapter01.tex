%************************************************
\chapter{Teoria}\label{ch:teoria}
%************************************************
\section{Introduzione}

La TDA si prefigge come obiettivo quello di trovare strutture complesse in un insieme di dati. Finora sono stati conseguiti risultati come la determinazione di nuove variabili che influenzano l'attività neurale \cite{Spreemann2015}, la classificazione di traiettorie in robotica \cite{Pokorny2014}, l'identificazione di nuovi tipi di cancro al seno \cite{Lum2013}, e molti altri.

La novità della TDA sta nel provare a catturare la \emph{forma} dei dati e, in questa, cercare proprietà topologiche interessanti che costituiscano un segnale anziché un rumore. L'idea fondamentale è la persistenza, cioé considerare i dati su tutte le scale e cercare quali segnali si mantengono in ampi intervalli di scala. Il modo principale per visualizzare queste proprietà sono i codici a barre, che saranno oggetto di questo capitolo.

\section{L'omologia persistente}
